\begin{tabular}{ %
       |>{\colleft}p{4cm}%
       |>{\colleft}p{8.5cm}|}
\hline
{\bf Communication model} & {\bf Information Exchange Specification Worksheet CM-2} \\
\hline
\hline
{\sc Transaction} &
%   Give the transaction identifier and the name of which this information
%   exchange specification is a part.
Transaction 2: Negotiate Observable
 \\
\hline
{\sc Agents involved} &
1. {\bf Sender}
%agent sending the information item(s)
CRA send a request for an obervation, and an explanation of that observation
\newline
2. {\bf Receiver}:
The hobbyist receives the request for an observation, and an explanation.
%agent receiving the information item(s)
\\
\hline
{\sc Information items} &
There is two information objects, the name of the obervation to be done, and an
explanation of that obervation. 
%   List all information items that are to be transmitted in this
%   transaction. This includes the (`core') information object the
%   transfer of which is the purpose of the transaction. However, it may contain
%   other, supporting, information items, that, for example, provide help
%   or explanation. For each information item, describe the following:
   \\
&  1. {\bf Role}: The name of the observation is core, while the explanation is
support.
%whether it is a core object, or a support item.
   \\
&  2. {\bf Form}: The name of the observation is a string. The explanation is
canned rich text.
     % the syntactic form in which it transmitted to
     % another agent , e.g., data string, canned text, a certain type of
     % diagram, 2D or 3D plot.
   \\
&  3. {\bf Medium}: The name of the observation can be selected in a menu. The
explanation can be shown in a text box.
     % the medium through which it is handled in the
     % agent-agent interaction, e.g., a pop-up window, navigation and
     % selection within a menu, command-line interface, human
     % intervention.
   \\
\hline
{\sc Message specifications} & \\
{\sc 1. REQUEST-OBSERVATION} &
   {\bf Communication type}: REQUEST\\
&  {\bf Content}: Request for some observation\\
&  {\bf From}: CRA\\
&  {\bf To}: The hobbyist\\
   \\
{\sc 2. OFFER-OBSERVATION} &
   {\bf Communication type}: OFFER\\
&  {\bf Content}: The Hobbyist wants to do a certain observation\\
&  {\bf From}: The hobbyist\\
&  {\bf To}: CRA\\
   \\
{\sc 3. DO-OBSERVATION} &
   {\bf Communication type}: ORDER\\
&  {\bf Content}: explanation and observation the hobbyist needs to make\\
&  {\bf From}: CRA\\
&  {\bf To}: The hobbyist\\
   \\
{\sc 4. REJECT-OBSERVATION-REQUEST} &
   {\bf Communication type}: REJECT-ta\\
&  {\bf Content}: Don't want to do this observation\\
&  {\bf From}: The hobbyist\\
&  {\bf To}: CRA\\
   \\
{\sc 5. REJECT-OBSERVATION-OFFER} &
   {\bf Communication type}: REJECT-td\\
&  {\bf Content}: Explanation why that observation is not needed\\
&  {\bf From}: CRA\\
&  {\bf To}: The hobbyist\\
   \\
\hline
\sc Control over messages &
    % Give, if necessary, a control specification over the messages
    % within the transaction. This can be done in pseudocode format or
    % in a state-transition diagram, similar to how the control over
    % transaction within the communication plan is specified. The
    % difference is just the level of detail.
    See figure 
   \\
\hline
\end{tabular}
