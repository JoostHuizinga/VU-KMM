% 1,2,5
\noindent
\begin{tabular}{%
       |>{\colleft}p{3cm}%
       |>{\colleft}p{8.5cm}|}
\hline
{\bf Communication model} &
   {\bf Transaction Description Worksheet CM-1} \\
\hline
\hline
\sc Transaction identifier/name &
	\emph{Transaction 1: Report complaint} \\

   %{\rm
   %A transaction is to be defined for each information object that is
   %output from some leaf task in the task model or in the knowledge
   %model (i.e., a transfer function), and that must be communicated to
   %another agent for use in its own tasks. The name must reflect, in a
   %user-understandable way, what is done with this information object
   %by the transaction. In addition to the name, give a brief
   %explanation here of the purpose of the transaction.
   %} 
\hline
\sc Information object &
	Transferring a complaint between the \emph{find complaint} and \emph{cover complaint} task. \\
   %{\rm
   %Indicate the (core) information object, and between which two
   %tasks it is to be transmitted.
   %} \\
\hline
\sc Agents involved &
	\emph{Car repair assistant}: receiving the complaint;\newline
	\emph{Hobbyist}: sending the complaint\\
   %{\rm
   %Indicate the agent that is sender of the information object,
   %and the agent that is receiving it.
   %} \\
\hline
\sc Communication plan &
	CRA communication plan\\

   %{\rm
   %Indicate the communication plan of which this transaction is a
   %component.
   %} \\
\hline
\sc Constraints &
	Before the transaction the car repair assistant must be ready to reiceve complaints\\

   %{\rm
   %Specify the requirements and (pre)conditions that must be fulfilled
   %so that the transaction can be carried out. Sometimes, it is also
   %useful to state post-conditions that are assumed to be valid after
   %the transaction.
   %} \\
\hline
\sc Information exchange specification &
	See worksheet CM-2 below.\\

   %{\rm
   %Transactions can have an internal structure, in that they consist
   %of several messages of different types, and/or handle additional
   %supporting information objects such as explanation or help items.
   %This is detailed in worksheet CM-2. At this point, only a reference
   %or pointer needs to be given to a later info exchange spec.
   %} \\
\hline
\end{tabular}


\noindent
\begin{tabular}{%
       |>{\colleft}p{3cm}%
       |>{\colleft}p{8.5cm}|}
\hline
{\bf Communication model} &
   {\bf Transaction Description Worksheet CM-1} \\
\hline
\hline
\sc Transaction identifier/name &
	\emph{Transaction 2: Propose hypothesis} \\

   %{\rm
   %A transaction is to be defined for each information object that is
   %output from some leaf task in the task model or in the knowledge
   %model (i.e., a transfer function), and that must be communicated to
   %another agent for use in its own tasks. The name must reflect, in a
   %user-understandable way, what is done with this information object
   %by the transaction. In addition to the name, give a brief
   %explanation here of the purpose of the transaction.
   %} 
\hline
\sc Information object &
	Transfering sets of hypothesis between the \emph{propose hypothesis} and \emph{select hypothesis} task. \\
   %{\rm
   %Indicate the (core) information object, and between which two
   %tasks it is to be transmitted.
   %} \\
\hline
\sc Agents involved &
	\emph{Hobbyist}: receiving the set of proposed hypothesis, sending a set of hypothesis (might be an empty set); \newline
	\emph{Car repair assistant}: sending a set of proposed hypothesis; receiving a set of hypothesis\\
   %{\rm
   %Indicate the agent that is sender of the information object,
   %and the agent that is receiving it.
   %} \\
\hline
\sc Communication plan &
	CRA communication plan\\

   %{\rm
   %Indicate the communication plan of which this transaction is a
   %component.
   %} \\
\hline
\sc Constraints &
	Before the transaction the car repair assistant must have a set of hypothesis ready. As a post condition one hypothesis has to be selected.\\

   %{\rm
   %Specify the requirements and (pre)conditions that must be fulfilled
   %so that the transaction can be carried out. Sometimes, it is also
   %useful to state post-conditions that are assumed to be valid after
   %the transaction.
   %} \\
\hline
\sc Information exchange specification &
	See worksheet CM-2 below.\\

   %{\rm
   %Transactions can have an internal structure, in that they consist
   %of several messages of different types, and/or handle additional
   %supporting information objects such as explanation or help items.
   %This is detailed in worksheet CM-2. At this point, only a reference
   %or pointer needs to be given to a later info exchange spec.
   %} \\
\hline
\end{tabular}

\noindent
\begin{tabular}{%
       |>{\colleft}p{3cm}%
       |>{\colleft}p{8.5cm}|}
\hline
{\bf Communication model} &
   {\bf Transaction Description Worksheet CM-1} \\
\hline
\hline
\sc Transaction identifier/name &
	\emph{Transaction 3: Negotiate observable} \\

   %{\rm
   %A transaction is to be defined for each information object that is
   %output from some leaf task in the task model or in the knowledge
   %model (i.e., a transfer function), and that must be communicated to
   %another agent for use in its own tasks. The name must reflect, in a
   %user-understandable way, what is done with this information object
   %by the transaction. In addition to the name, give a brief
   %explanation here of the purpose of the transaction.
   %} 
\hline
\sc Information object &
	Transferring observation instructions between the \emph{specify observable} and \emph{determine whether going to do observation} task. \\
   %{\rm
   %Indicate the (core) information object, and between which two
   %tasks it is to be transmitted.
   %} \\
\hline
\sc Agents involved &
	\emph{Car repair assistant}: sending observation instructions;\newline
	\emph{Hobbyist}: receiving observation instructions\\
   %{\rm
   %Indicate the agent that is sender of the information object,
   %and the agent that is receiving it.
   %} \\
\hline
\sc Communication plan &
	CRA communication plan\\

   %{\rm
   %Indicate the communication plan of which this transaction is a
   %component.
   %} \\
\hline
\sc Constraints &
	Before the transaction the car repair assistant must have a set of observables ready. As a post condition one observable must be excepted.\\

   %{\rm
   %Specify the requirements and (pre)conditions that must be fulfilled
   %so that the transaction can be carried out. Sometimes, it is also
   %useful to state post-conditions that are assumed to be valid after
   %the transaction.
   %} \\
\hline
\sc Information exchange specification &
	See worksheet CM-2 below.\\

   %{\rm
   %Transactions can have an internal structure, in that they consist
   %of several messages of different types, and/or handle additional
   %supporting information objects such as explanation or help items.
   %This is detailed in worksheet CM-2. At this point, only a reference
   %or pointer needs to be given to a later info exchange spec.
   %} \\
\hline
\end{tabular}


\noindent
\begin{tabular}{%
       |>{\colleft}p{3cm}%
       |>{\colleft}p{8.5cm}|}
\hline
{\bf Communication model} &
   {\bf Transaction Description Worksheet CM-1} \\
\hline
\hline
\sc Transaction identifier/name &
	\emph{Transaction 4: Report observable} \\

   %{\rm
   %A transaction is to be defined for each information object that is
   %output from some leaf task in the task model or in the knowledge
   %model (i.e., a transfer function), and that must be communicated to
   %another agent for use in its own tasks. The name must reflect, in a
   %user-understandable way, what is done with this information object
   %by the transaction. In addition to the name, give a brief
   %explanation here of the purpose of the transaction.
   %} 
\hline
\sc Information object &
	Transferring observation result between the \emph{perform observation} and \emph{verify} task. \\
   %{\rm
   %Indicate the (core) information object, and between which two
   %tasks it is to be transmitted.
   %} \\
\hline
\sc Agents involved &
	\emph{Hobbyist}: sending observation result;			\newline
	\emph{Car repair assistant}: receiving observation result	\\
   %{\rm
   %Indicate the agent that is sender of the information object,
   %and the agent that is receiving it.
   %} \\
\hline
\sc Communication plan &
	CRA communication plan\\

   %{\rm
   %Indicate the communication plan of which this transaction is a
   %component.
   %} \\
\hline
\sc Constraints &
	Before the transaction the hobbyist must have carried out the observation instructions and remembered there results.\\
   %{\rm
   %Specify the requirements and (pre)conditions that must be fulfilled
   %so that the transaction can be carried out. Sometimes, it is also
   %useful to state post-conditions that are assumed to be valid after
   %the transaction.
   %} \\
\hline
\sc Information exchange specification &
	See worksheet CM-2 below.\\

   %{\rm
   %Transactions can have an internal structure, in that they consist
   %of several messages of different types, and/or handle additional
   %supporting information objects such as explanation or help items.
   %This is detailed in worksheet CM-2. At this point, only a reference
   %or pointer needs to be given to a later info exchange spec.
   %} \\
\hline
\end{tabular}

\noindent
\begin{tabular}{|>{\colleft}p{3cm}|>{\colleft}p{8.5cm}|}
\hline
{\bf Communication model} &
   {\bf Transaction Description Worksheet CM-1} \\
\hline
\hline
\sc Transaction identifier/name &
	\emph{Transaction 5: Report hypothesis} \\

   %{\rm
   %A transaction is to be defined for each information object that is
   %output from some leaf task in the task model or in the knowledge
   %model (i.e., a transfer function), and that must be communicated to
   %another agent for use in its own tasks. The name must reflect, in a
   %user-understandable way, what is done with this information object
   %by the transaction. In addition to the name, give a brief
   %explanation here of the purpose of the transaction.
   %} 
\hline
\sc Information object &
	Transferring hypothesis result between the \emph{verify} and \emph{repair car} task. \\
   %{\rm
   %Indicate the (core) information object, and between which two
   %tasks it is to be transmitted.
   %} \\
\hline
\sc Agents involved &
	\emph{Car repair assistant}: sending hypothesis;	\newline
	\emph{Hobbyist}: receiving hypothesis			\\
   %{\rm
   %Indicate the agent that is sender of the information object,
   %and the agent that is receiving it.
   %} \\
\hline
\sc Communication plan &
	CRA communication plan\\

   %{\rm
   %Indicate the communication plan of which this transaction is a
   %component.
   %} \\
\hline
\sc Constraints &
	Before the transaction the car repair assistant must have either no observations left or he has one or less hypothesis left.\\
   %{\rm
   %Specify the requirements and (pre)conditions that must be fulfilled
   %so that the transaction can be carried out. Sometimes, it is also
   %useful to state post-conditions that are assumed to be valid after
   %the transaction.
   %} \\
\hline
\sc Information exchange specification &
	See worksheet CM-2 below.\\

   %{\rm
   %Transactions can have an internal structure, in that they consist
   %of several messages of different types, and/or handle additional
   %supporting information objects such as explanation or help items.
   %This is detailed in worksheet CM-2. At this point, only a reference
   %or pointer needs to be given to a later info exchange spec.
   %} \\
\hline
\end{tabular}