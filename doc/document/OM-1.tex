%%%%%%%%%%%%%%%%%%%%%%%%%%%%%%%%%%%%%%%
% OM-1
\noindent
\begin{tabular}{|>{\colleft}p{3cm}|>{\colleft}p{10cm}|}
\hline
{\bf Organization Model} &
   {\bf Problems and Opportunities Worksheet OM-1} \\
\hline
\hline
{\sc Problems and opportunities} &
% Make a shortlist of perceived problems and opportunities, based on
% interviews, brainstorm and visioning meetings, discussions with
% managers, etc.
Hobbyists currently experience problems with the availability of documentation,
reasoning about complex, partly unknown technical systems and gaps in their own
knowledge.
\\ \hline
{\sc Organizational context} &
% Indicate in a concise manner key features of the wider
% organizational context, so as to put the listed opportunities and
% problems into proper perspective. Important features to consider
%  are:
% 1. Mission, vision, goals of the organization
% 2. Important external factors the organization has to deal with
% 3. Strategy of the organization
% 4. Its value chain and the major value drivers
The goal of the hobbyist is to repair his or her car. He wants to have fun and
learn about cars. He might also save money because he does not have to go to a
garage or he might be better than a garage in repairing his car because he has 
more specific knowledge about his car.

The car needs to be in a good enough shape to pass the APK. Other users of the
car require the car to be available. External factors like work and family restrict the time and resources available to repair in ways that are uncontrollable or unexpected.
\\ \hline

{\sc Solutions} &
% List possible solutions for the perceived problems and
% opportunities, as suggested by the interviews and discussions held,
% and the above features of the organizational context.
Possible solutions are:
\begin{itemize}
	\item An information retrieval system to find documentation about car repair and
specific cars on the Internet. 
	\item A knowledge system holding knowledge about car repair, reasoning about that
knowledge and using it to assist the hobbyist in repairing his car. 
	\item Educating the car hobbyist about car repair.
\end{itemize}
\\ \hline
\end{tabular}
%%%%%%%%%%%%%%%%%%%%%%%%%%%%%%%%%%%%%%%
