\section{Conclusion}
\subsection{The process}
The process of creating the knowledge based system went well in general.

There was a problem with, unsurprisingly, the contact with our expert. Although the expert, a student car technician, was chosen because he was more easily accessible then the average car mechanic it was still difficult to get an appointment. Because of this the initial domain schema was eventually created without the help of the expert. Also, since we initially waited for an appointment there was a great delay in our project. The information that was eventually obtained through our expert could be added rather nicely to our domain knowledge, but this ordering undoubtedly influenced the final model.

We used most of the models and methods of CommonKADS during the project. The organization and agent
model is useful to define the boundary and context of the system. As the environment
of the system is very small, it only works together with the car hobbyist, we
quickly finished those. With the task model we could better define and narrow
down the task of the system. The knowledge model and the communication model
are very useful. They form the foundation of our implementation.

The knowledge
model makes clear what rules and inferences should be used by the system. It
also created problems which have disappeared when we implemented it. We struggled
with the
difference between inferences and rules types, designing a rule type for the
cover inference separately from a rule type for the verify and specify step. This
proved unnecessary when we implemented the rules.

The CRA needs to work together with the car hobbyist. Communication is essential
for the system. The communication model is great to model this. Creating this
model was time-consuming, but proved worthwhile.

On a first attempt we did try to make a design model based on the model view controller paradigm.
It was very unclear how this should be done. With little time left and good
preceding models we decided to just start the implementation based on the
knowledge and communication model. We implemented the causal rules and component
knowledge from the knowledge model in Jess. These rules were used in the
inferences that we also implemented in Jess. The control of the program is based
on the communication plan and dialog diagram of the communication model and
implemented in Java.

On a second attempt we documented what we implemented. From the activity diagram and sequence diagrams in de design model a model view controller patern emerged. Now it was clear how the design should proceed. Based on this new design we changed out program. Classes for hypothesis, components, states, observables and findings were added. The single application was split up between a controller, a view and a model class. At first this enabled us to add the try to repair step in the design and program. Second it allowed us move the logic to deal with hypothesis with multiple faulty components from Jess to the Java class Hypothesis, leading to improvements in dealing with multiple hypothesis. We didn't need to change the core logic of the program. Third it enables improvements in the user interaction.

\subsection{The final result}
The Car Repair Assistant performs its tasks reasonably well. It gives all possible causes and it will slowly reduce the number of possible causes by querying the user about various observables. The user can also steers this process by giving suggestion about which causes should be examined next. The system is also very flexible and using it with an other car, or even an other electrical system is very simple. Nevertheless, there are a lot of possible improvements.

An other important point that could be improved upon is the method that the CRA uses to suggest hypothesis and observables. Right now it just choses one with static preferences. It preferes hypothesis without wires and observables that don't observe the current hypothesis. It would be a lot better if these would be selected based upon the ease of obtaining the observations or based upon the information gain.

A last obvious point of improvement would be in the interface. This system would be much more practical if it worked with a transparent picture of a car that depicted all the components and their locations. This would make the tool a lot easier to use for the users.

Unfortunately there was little time to create the implementation of the final Car Repair Assistant. This is part due to the delay at the start of our project and part due to the fact that two months is not very long to create a fully functional knowledge system. The extra time provided to us to improve our work enables us to improve the design model and documentation, and to improve the visible structure of the program to confirm with the design.
